% Created 2022-04-04 Mon 15:02
% Intended LaTeX compiler: pdflatex
\documentclass[11pt]{article}
\usepackage[utf8]{inputenc}
\usepackage[T1]{fontenc}
\usepackage{graphicx}
\usepackage{grffile}
\usepackage{longtable}
\usepackage{wrapfig}
\usepackage{rotating}
\usepackage[normalem]{ulem}
\usepackage{amsmath}
\usepackage{textcomp}
\usepackage{amssymb}
\usepackage{capt-of}
\usepackage{hyperref}
% org has these:
% \usepackage[T1]{fontenc}
% \usepackage[utf8]{inputenc}

\usepackage[letterpaper,margin=2cm]{geometry}
%\usepackage{fullpage}
% a better font family
\usepackage{lmodern} % for sans serif, the fonts below overrride this
\usepackage{eulervm}
%\usepackage{newpxtext}
% \usepackage{newpxmath}
\usepackage{mathpazo}
%\usepackage{beton}
%\usepackage[default]{comicneue}

\usepackage{xcolor}

% \usepackage{amssymb}
\usepackage{amsthm}
% \usepackage{amsmath}
\usepackage{amstext}
% math tools for amsmath
\usepackage{mathtools}
% ceiling and floor symbols
\DeclarePairedDelimiter\ceil{\lceil}{\rceil}
\DeclarePairedDelimiter\floor{\lfloor}{\rfloor}

\usepackage{braket} % to enter bra-ket notation - easily
\usepackage{wasysym}

% to draw quantum circuits
\usepackage{qcircuit}

% for more enumeration style options
% \usepackage{enumerate}
\usepackage{enumitem}

% for references, org-mode already has these
% \usepackage{varioref}
% \usepackage[hidelinks]{hyperref}
\usepackage{cleveref} % for automatically entering reference types (theorem, figure, section etc.)

\usepackage{titling, titlesec}
\usepackage{slantsc} 

% bibliography style
\bibliographystyle{acm}

% theorem types
\newtheorem{thm}{Theorem}
\newtheorem{cor}[thm]{Corollary}
\newtheorem{lem}[thm]{Lemma}
% theorem types' names for cleveref
\crefname{thm}{theorem}{theorems}
\crefname{lem}{lemma}{lemmas}
\crefname{cor}{corollary}{corollaries}
\Crefname{thm}{Theorem}{Theorems}
\Crefname{lem}{Lemma}{Lemmas}
\Crefname{cor}{Corollary}{Corollaries}

\newcommand{\draftnote}[1]{\textcolor{red}{#1}}

\newcommand*{\QED}{\null\hfill\qedsymbol}

% I mistype texttt sometimes
\newcommand{\textt}{\texttt}
% ease to enter some functions in math mode
\newcommand{\fn}{\mathrm}
\newcommand{\bigO}{\fn{O}}
\newcommand{\argmax}{\fn{argmax}}
\newcommand{\ii}{\imath}
\newcommand{\qre}{q^\circ}
\newcommand{\qim}{q^\ast}
\newcommand{\transop}{\mathcal{E}}
\newcommand{\lcm}{\fn{lcm}}
\newcommand{\Str}{\mathbb{S}}
\newcommand{\Sign}{\Str}
\newcommand{\Concrete}{\mathcal{P}(\Sigma^*)}
\newcommand{\true}{\mathrm{true}}
\newcommand{\false}{\mathrm{false}}
\newcommand{\conj}{\overline}
\newcommand{\otherwise}{\textrm{ otherwise}}
\newcommand{\where}[1]{\textrm{ where {#1}}}

% norms and absolute values
\newcommand{\abs}[1]{\left\vert{}#1\right\vert}
\newcommand{\norm}[1]{\left\Vert{}#1\right\Vert}
\newcommand{\fnorm}[1]{\norm{#1}_F}
\newcommand{\pnorm}[2]{\norm{#1}_{#2}}
\newcommand{\supx}[1]{\sup_{#1 \neq 0}}
\newcommand{\supn}[2]{\sup_{\norm{#1}_{#2} = 1}}

% trace
\DeclareMathOperator{\tr}{Tr}

%\newcommand{\span}{\fn{span}}
\newcommand{\cols}{\fn{cols}}
\newcommand{\range}{\mathcal{R}}
\newcommand{\rank}{\fn{rank}}
\newcommand{\eye}{\fn{I}}

\newcommand{\diag}[1]{\fn{diag}\left\{#1\right\}}

% small matrices
\newenvironment{mat}[1]{\left(\begin{array}{#1}}{\end{array}\right)}
\newcommand{\matone}[1]{\begin{mat}{l}#1\end{mat}}
\newcommand{\mattwo}[1]{\begin{mat}{ll}#1\end{mat}}
\newcommand{\matthree}[1]{\begin{mat}{lll}#1\end{mat}}
\newcommand{\matfour}[1]{\begin{mat}{llll}#1\end{mat}}

\newcommand{\eyetwo}{\mattwo{1 & 0 \\ 0 & 1}}

% extra functions
\DeclareMathOperator{\fid}{F} % fidelity
\DeclareMathOperator{\vecOf}{vec} % vectorization

% special sets
\newcommand{\mset}{\mathbb}
\newcommand{\reals}{\mset{R}}
\newcommand{\realmat}[2]{\reals^{#1 \times #2}}
\newcommand{\complex}{\mset{C}}
\newcommand{\complexmat}[2]{\complex^{#1 \times #2}}
\newcommand{\zahlen}{\mset{Z}}
\newcommand{\ints}{\zahlen}
\newcommand{\nats}{\mset{N}}
% group, field etc.
\newcommand{\field}{\mset{F}}

% questions
\newenvironment{questions}
{\begin{enumerate}[label={Question \arabic*},wide,font=\bf]}
  {\end{enumerate}}

\newcommand{\question}{\item \mbox{} \\}
\newcommand{\Question}{\newpage \item \mbox{} \\}

% notation
\newcommand{\notation}[1]{
\textbf{Notation.} #1
}

% quantum gates
\newcommand{\qgate}[1]{\mathrm{#1}}
\newcommand{\SWAP}{\qgate{SWAP}}
\newcommand{\SSWAP}{\sqrt{\SWAP}}

% spaces
\newcommand{\lin}{\mathcal{L}}
\newcommand{\unitary}[1]{\mathcal{U}(#1)}
\newcommand{\density}{\mathcal{D}}
\newcommand{\stdbasis}{\mathcal{B}}

\author{Instructor: Mehmet Emre}
\date{CS 32 Spring '22}
\title{Homework 3: Object-Oriented Design}
\hypersetup{
 pdfauthor={Instructor: Mehmet Emre},
 pdftitle={Homework 3: Object-Oriented Design},
 pdfkeywords={},
 pdfsubject={},
 pdfcreator={Emacs 27.2 (Org mode 9.4.4)}, 
 pdflang={English}}
\begin{document}

\maketitle
\textbf{Due: 4/13 12:30pm} \\ 
\textbf{Name \& Perm \#:} \\ 
\textbf{Homework buddy (leave blank if you worked alone):}

\textbf{Reading: Object Oriented Design, PS 10.2}


\section{}
\label{sec:org311bec3}
According to Savitch in PS, the scope resolution operator \texttt{::} and
the dot operator \texttt{.} have a similar purpose, but there is a major
difference between them.
\begin{enumerate}
\item (5 pts) What do they have in common?
\begin{description}
    \item[Answer:] .\\
    They both are used to tell what a member function is a member of.
  \end{description}
\vspace{2em}
\item (5 pts) What is the difference between them?
\begin{description}
    \item[Answer:] .\\
    The dot operator is used on actual objects, while the seperator is used on classes.
  \end{description}
\vspace{6em}
\end{enumerate}


In a class for an object representing a video game character: 
\section{(5 pts)}
\label{sec:orge0fd38b}
Would \texttt{setHealth} be an accessor or a mutator function?
    Circle one: accessor \(\quad\) mutator
    \begin{description}
        \item[Answer:] .\\
        mutator
    \end{description}
\section{(5 pts)}
\label{sec:org413e13f}
Would \texttt{getName} be an accessor or a mutator function?
    Circle one: accessor \(\quad\) mutator
    \begin{description}
        \item[Answer:] .\\
        accessor
    \end{description}

\newpage

\section{(5 pts)}
\label{sec:org6372af0}
What does the term \emph{encapsulation} refer to?
\begin{description}
    \item[Answer:] .\\
    Encapsulation refers to combining variables and functions under a single class.
\end{description}
\vspace{6em}

\section{(5 pts)}
\label{sec:org66c5004}
According to Savitch in PS, it is normal practice to make member
functions private under what circumstances?
\begin{description}
    \item[Answer:] .\\
    When you want a member function accessible to other functions in that class but do not want them to be accessible from outside (callable from objects of the class). This can be useful if a function is a helper function that is only expected to be used by other functions inside that class.
\end{description}
\newpage

\section{}
\label{sec:orgeb87943}

On p. 579 and then again on p. 586 Savitch drives home a point about
the syntax of invoking a constructor--what you SHOULD do when invoking
a no-arg constructor, and what you should NOT do. Though he doesn't
mention it, this is a "trap" that many C++ learners fall into if they
learned Java first, because this is a place where C++ and Java syntax
differ significantly.

Suppose you have a no-arg constructor for a class \texttt{Student}.   

\begin{enumerate}
\item (5 pts) What is the correct syntax to declare a local variable inside a function or method called \texttt{s} that is of type \texttt{Student}, and is created with the no-arg constructor?
\begin{description}
    \item[Answer:]
    \begin{verbatim}
        Student s;
    \end{verbatim}
\end{description}
\item (5 pts) What is the "wrong" syntax for doing that same thing that Savitch specifically warns \emph{against} doing?
\begin{description}
    \item[Answer:]
    \begin{verbatim}
        Student s();
    \end{verbatim}
\end{description}
\item (5 pts) In one of the passages where this is explained, Savitch indicates WHY this other syntax is wrong if your intention is to make \texttt{s} an instance of class \texttt{Student}.  This syntax actually has a completely different meaning in C++. What does this alternate syntax mean--that is, what would \texttt{s} be declared to be under this alternate syntax?
\begin{description}
    \item[Answer:] .\\
    The alternate syntax is used for function declaration in C++.
\end{description}

\item (5 pts) Suppose you have a class \texttt{Student} with private data members as shown below. You could write a constructor like this:
\end{enumerate}
\begin{verbatim}
class Student {
    private:
       int perm_;
       std::string name_;
  }

  Student::Student(int perm, std::string name) {
     perm_ = perm;
     name_ = name;
  }
\end{verbatim}

However, there is an alternate way to initialize data members in the
so called "constructor initialization section" described in Section
10.2 of PS. Rewrite this constructor so that the body is an empty set
of braces, and the code is moved to the so-called "constructor
initialization section".\footnote{Note that I am \emph{not} referring to C++11's
"constructor delegation" described on p. 587-588; this is a feature
that has been in C++ for many years before C++11, and is described
somewhere before p. 581-584.}

\begin{description}
    \item[Answer:]
    \begin{verbatim}
        Student::Student(int perm, std::string name) : perm_(perm), name_(name) {}
    \end{verbatim}
\end{description}
\end{document}