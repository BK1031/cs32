% Created 2022-04-30 Sat 17:49
% Intended LaTeX compiler: pdflatex
\documentclass[11pt]{article}
\usepackage[utf8]{inputenc}
\usepackage[T1]{fontenc}
\usepackage{graphicx}
\usepackage{longtable}
\usepackage{wrapfig}
\usepackage{rotating}
\usepackage[normalem]{ulem}
\usepackage{amsmath}
\usepackage{amssymb}
\usepackage{capt-of}
\usepackage{hyperref}
% org has these:
% \usepackage[T1]{fontenc}
% \usepackage[utf8]{inputenc}

\usepackage[letterpaper,margin=2cm]{geometry}
%\usepackage{fullpage}
% a better font family
\usepackage{lmodern} % for sans serif, the fonts below overrride this
\usepackage{eulervm}
%\usepackage{newpxtext}
% \usepackage{newpxmath}
\usepackage{mathpazo}
%\usepackage{beton}
%\usepackage[default]{comicneue}

\usepackage{xcolor}

% \usepackage{amssymb}
\usepackage{amsthm}
% \usepackage{amsmath}
\usepackage{amstext}
% math tools for amsmath
\usepackage{mathtools}
% ceiling and floor symbols
\DeclarePairedDelimiter\ceil{\lceil}{\rceil}
\DeclarePairedDelimiter\floor{\lfloor}{\rfloor}

\usepackage{braket} % to enter bra-ket notation - easily
\usepackage{wasysym}

% to draw quantum circuits
\usepackage{qcircuit}

% for more enumeration style options
% \usepackage{enumerate}
\usepackage{enumitem}

% for references, org-mode already has these
% \usepackage{varioref}
% \usepackage[hidelinks]{hyperref}
\usepackage{cleveref} % for automatically entering reference types (theorem, figure, section etc.)

\usepackage{titling, titlesec}
\usepackage{slantsc} 

% bibliography style
\bibliographystyle{acm}

% theorem types
\newtheorem{thm}{Theorem}
\newtheorem{cor}[thm]{Corollary}
\newtheorem{lem}[thm]{Lemma}
% theorem types' names for cleveref
\crefname{thm}{theorem}{theorems}
\crefname{lem}{lemma}{lemmas}
\crefname{cor}{corollary}{corollaries}
\Crefname{thm}{Theorem}{Theorems}
\Crefname{lem}{Lemma}{Lemmas}
\Crefname{cor}{Corollary}{Corollaries}

\newcommand{\draftnote}[1]{\textcolor{red}{#1}}

\newcommand*{\QED}{\null\hfill\qedsymbol}

% I mistype texttt sometimes
\newcommand{\textt}{\texttt}
% ease to enter some functions in math mode
\newcommand{\fn}{\mathrm}
\newcommand{\bigO}{\fn{O}}
\newcommand{\argmax}{\fn{argmax}}
\newcommand{\ii}{\imath}
\newcommand{\qre}{q^\circ}
\newcommand{\qim}{q^\ast}
\newcommand{\transop}{\mathcal{E}}
\newcommand{\lcm}{\fn{lcm}}
\newcommand{\Str}{\mathbb{S}}
\newcommand{\Sign}{\Str}
\newcommand{\Concrete}{\mathcal{P}(\Sigma^*)}
\newcommand{\true}{\mathrm{true}}
\newcommand{\false}{\mathrm{false}}
\newcommand{\conj}{\overline}
\newcommand{\otherwise}{\textrm{ otherwise}}
\newcommand{\where}[1]{\textrm{ where {#1}}}

% norms and absolute values
\newcommand{\abs}[1]{\left\vert{}#1\right\vert}
\newcommand{\norm}[1]{\left\Vert{}#1\right\Vert}
\newcommand{\fnorm}[1]{\norm{#1}_F}
\newcommand{\pnorm}[2]{\norm{#1}_{#2}}
\newcommand{\supx}[1]{\sup_{#1 \neq 0}}
\newcommand{\supn}[2]{\sup_{\norm{#1}_{#2} = 1}}

% trace
\DeclareMathOperator{\tr}{Tr}

%\newcommand{\span}{\fn{span}}
\newcommand{\cols}{\fn{cols}}
\newcommand{\range}{\mathcal{R}}
\newcommand{\rank}{\fn{rank}}
\newcommand{\eye}{\fn{I}}

\newcommand{\diag}[1]{\fn{diag}\left\{#1\right\}}

% small matrices
\newenvironment{mat}[1]{\left(\begin{array}{#1}}{\end{array}\right)}
\newcommand{\matone}[1]{\begin{mat}{l}#1\end{mat}}
\newcommand{\mattwo}[1]{\begin{mat}{ll}#1\end{mat}}
\newcommand{\matthree}[1]{\begin{mat}{lll}#1\end{mat}}
\newcommand{\matfour}[1]{\begin{mat}{llll}#1\end{mat}}

\newcommand{\eyetwo}{\mattwo{1 & 0 \\ 0 & 1}}

% extra functions
\DeclareMathOperator{\fid}{F} % fidelity
\DeclareMathOperator{\vecOf}{vec} % vectorization

% special sets
\newcommand{\mset}{\mathbb}
\newcommand{\reals}{\mset{R}}
\newcommand{\realmat}[2]{\reals^{#1 \times #2}}
\newcommand{\complex}{\mset{C}}
\newcommand{\complexmat}[2]{\complex^{#1 \times #2}}
\newcommand{\zahlen}{\mset{Z}}
\newcommand{\ints}{\zahlen}
\newcommand{\nats}{\mset{N}}
% group, field etc.
\newcommand{\field}{\mset{F}}

% questions
\newenvironment{questions}
{\begin{enumerate}[label={Question \arabic*},wide,font=\bf]}
  {\end{enumerate}}

\newcommand{\question}{\item \mbox{} \\}
\newcommand{\Question}{\newpage \item \mbox{} \\}

% notation
\newcommand{\notation}[1]{
\textbf{Notation.} #1
}

% quantum gates
\newcommand{\qgate}[1]{\mathrm{#1}}
\newcommand{\SWAP}{\qgate{SWAP}}
\newcommand{\SSWAP}{\sqrt{\SWAP}}

% spaces
\newcommand{\lin}{\mathcal{L}}
\newcommand{\unitary}[1]{\mathcal{U}(#1)}
\newcommand{\density}{\mathcal{D}}
\newcommand{\stdbasis}{\mathcal{B}}

\author{Instructor: Mehmet Emre}
\date{CS 32 Spring '22}
\title{Homework 9: Hashing}
\hypersetup{
 pdfauthor={Instructor: Mehmet Emre},
 pdftitle={Homework 9: Hashing},
 pdfkeywords={},
 pdfsubject={},
 pdfcreator={Emacs 28.1 (Org mode 9.5.2)}, 
 pdflang={English}}
\begin{document}

\maketitle
\textbf{Due: 5/11 12:30pm} \\ 
\vspace{1em}
\textbf{Name \& Perm \#: Bharat Kathi (5938444)} \\ 
\textbf{Homework buddy (leave blank if you worked alone):}

\textbf{Reading:} DS 12.2

\section{}
\label{sec:org07b2aca}
From DS (12.2): There is a technique known as "double hashing".
\begin{enumerate}
\item (4 pts) What is the problem that double hashing is designed to solve?
\textbf{Briefly explain.}
\begin{description}
    \item[Answer:] Double hashing is used to resolve collisions that occur in hash tables. This is when a hash table's hash function returns the same index for two different keys.
\end{description}
\vspace{4em}
\item (4 pts) How does double hashing solve that problem? \textbf{Briefly explain.}
\begin{description}
    \item[Answer:] Double hashing takes care of this by the addition of another hash function. This second function is used to calculate an offset value in case the index generated by the first function causes a collision. This offset value can then be added to the first function's returned index to create a new index without collision to store the data.
\end{description}
\vspace{4em}
\end{enumerate}
\section{}
\label{sec:orgbd9e3a4}
From DS 12.2: The abstract data type (ADT) known as a "dictionary" provides a
way to look up keys and find values. We define these abstract operations for
the ADT:
\begin{description}
\item[{lookup(key)}] returns either value, or an indication that the value is not
present
\item[{remove(key)}] removes the item with that key (if it exists)
\item[{insert(key,value)}] inserts the new key, and the value (or replaces the
value if it is already present)
\end{description}

 We could implement a dictionary by having an array of (key,value) pairs,
sorted by key. We could use binary search for lookup. We'd have to figure out a
way to put new values into the sorted list, and remove values. Or we could use
hashing, as described in DS 12.2
(question continues on the next page)

\begin{enumerate}
\item (2 pts) If we use binary search on a sorted array, what is the worst case
time for lookup(key) in terms of big-O? (No explanation needed; just state
the answer)
\begin{description}
    \item[Answer:] O(log n)
\end{description}
\item (4 pts) If we use binary search on a sorted array, what is the worst case
big-O time for remove(key)? This time, \emph{briefly} explain your
answer.
\begin{description}
    \item[Answer:] O(log n). This is since the binary search keeps splitting the list into halfs until the search key is found.
\end{description}
\item (2 pts) If we use hash tables instead, what is the worst case big-O time
for lookup(key)? Just state it.
\begin{description}
    \item[Answer:] O(n)
\end{description}
\item (2 pts) Is the worst case for lookup(key) for hash tables better or worse
than the binary search approach?
\begin{description}
    \item[Answer:] The worst case time for the hash table is worse than that of the binary search approach.
\end{description}
\item (2 pts) If we use hash tables instead, what is the average (expected) case
big-O time for lookup(key) assuming no collisions? Just state it.
\begin{description}
    \item[Answer:] O(1)
\end{description}
\item (2 pts) Is the average (expected) case for lookup(key) for hash tables
better or worse than the binary search approach?
\begin{description}
    \item[Answer:] The average time for the hash table is better than that of the binary search approach.
\end{description}
\end{enumerate}
\end{document}