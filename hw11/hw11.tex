% Created 2021-11-03 Wed 14:30
% Intended LaTeX compiler: pdflatex
\documentclass[11pt]{article}
\usepackage[utf8]{inputenc}
\usepackage[T1]{fontenc}
\usepackage{graphicx}
\usepackage{grffile}
\usepackage{longtable}
\usepackage{wrapfig}
\usepackage{rotating}
\usepackage[normalem]{ulem}
\usepackage{amsmath}
\usepackage{textcomp}
\usepackage{amssymb}
\usepackage{capt-of}
\usepackage{hyperref}
% org has these:
% \usepackage[T1]{fontenc}
% \usepackage[utf8]{inputenc}

\usepackage[letterpaper,margin=2cm]{geometry}
%\usepackage{fullpage}
% a better font family
\usepackage{lmodern} % for sans serif, the fonts below overrride this
\usepackage{eulervm}
%\usepackage{newpxtext}
% \usepackage{newpxmath}
\usepackage{mathpazo}
%\usepackage{beton}
%\usepackage[default]{comicneue}

\usepackage{xcolor}

% \usepackage{amssymb}
\usepackage{amsthm}
% \usepackage{amsmath}
\usepackage{amstext}
% math tools for amsmath
\usepackage{mathtools}
% ceiling and floor symbols
\DeclarePairedDelimiter\ceil{\lceil}{\rceil}
\DeclarePairedDelimiter\floor{\lfloor}{\rfloor}

\usepackage{braket} % to enter bra-ket notation - easily
\usepackage{wasysym}

% to draw quantum circuits
\usepackage{qcircuit}

% for more enumeration style options
% \usepackage{enumerate}
\usepackage{enumitem}

% for references, org-mode already has these
% \usepackage{varioref}
% \usepackage[hidelinks]{hyperref}
\usepackage{cleveref} % for automatically entering reference types (theorem, figure, section etc.)

\usepackage{titling, titlesec}
\usepackage{slantsc} 

% bibliography style
\bibliographystyle{acm}

% theorem types
\newtheorem{thm}{Theorem}
\newtheorem{cor}[thm]{Corollary}
\newtheorem{lem}[thm]{Lemma}
% theorem types' names for cleveref
\crefname{thm}{theorem}{theorems}
\crefname{lem}{lemma}{lemmas}
\crefname{cor}{corollary}{corollaries}
\Crefname{thm}{Theorem}{Theorems}
\Crefname{lem}{Lemma}{Lemmas}
\Crefname{cor}{Corollary}{Corollaries}

\newcommand{\draftnote}[1]{\textcolor{red}{#1}}

\newcommand*{\QED}{\null\hfill\qedsymbol}

% I mistype texttt sometimes
\newcommand{\textt}{\texttt}
% ease to enter some functions in math mode
\newcommand{\fn}{\mathrm}
\newcommand{\bigO}{\fn{O}}
\newcommand{\argmax}{\fn{argmax}}
\newcommand{\ii}{\imath}
\newcommand{\qre}{q^\circ}
\newcommand{\qim}{q^\ast}
\newcommand{\transop}{\mathcal{E}}
\newcommand{\lcm}{\fn{lcm}}
\newcommand{\Str}{\mathbb{S}}
\newcommand{\Sign}{\Str}
\newcommand{\Concrete}{\mathcal{P}(\Sigma^*)}
\newcommand{\true}{\mathrm{true}}
\newcommand{\false}{\mathrm{false}}
\newcommand{\conj}{\overline}
\newcommand{\otherwise}{\textrm{ otherwise}}
\newcommand{\where}[1]{\textrm{ where {#1}}}

% norms and absolute values
\newcommand{\abs}[1]{\left\vert{}#1\right\vert}
\newcommand{\norm}[1]{\left\Vert{}#1\right\Vert}
\newcommand{\fnorm}[1]{\norm{#1}_F}
\newcommand{\pnorm}[2]{\norm{#1}_{#2}}
\newcommand{\supx}[1]{\sup_{#1 \neq 0}}
\newcommand{\supn}[2]{\sup_{\norm{#1}_{#2} = 1}}

% trace
\DeclareMathOperator{\tr}{Tr}

%\newcommand{\span}{\fn{span}}
\newcommand{\cols}{\fn{cols}}
\newcommand{\range}{\mathcal{R}}
\newcommand{\rank}{\fn{rank}}
\newcommand{\eye}{\fn{I}}

\newcommand{\diag}[1]{\fn{diag}\left\{#1\right\}}

% small matrices
\newenvironment{mat}[1]{\left(\begin{array}{#1}}{\end{array}\right)}
\newcommand{\matone}[1]{\begin{mat}{l}#1\end{mat}}
\newcommand{\mattwo}[1]{\begin{mat}{ll}#1\end{mat}}
\newcommand{\matthree}[1]{\begin{mat}{lll}#1\end{mat}}
\newcommand{\matfour}[1]{\begin{mat}{llll}#1\end{mat}}

\newcommand{\eyetwo}{\mattwo{1 & 0 \\ 0 & 1}}

% extra functions
\DeclareMathOperator{\fid}{F} % fidelity
\DeclareMathOperator{\vecOf}{vec} % vectorization

% special sets
\newcommand{\mset}{\mathbb}
\newcommand{\reals}{\mset{R}}
\newcommand{\realmat}[2]{\reals^{#1 \times #2}}
\newcommand{\complex}{\mset{C}}
\newcommand{\complexmat}[2]{\complex^{#1 \times #2}}
\newcommand{\zahlen}{\mset{Z}}
\newcommand{\ints}{\zahlen}
\newcommand{\nats}{\mset{N}}
% group, field etc.
\newcommand{\field}{\mset{F}}

% questions
\newenvironment{questions}
{\begin{enumerate}[label={Question \arabic*},wide,font=\bf]}
  {\end{enumerate}}

\newcommand{\question}{\item \mbox{} \\}
\newcommand{\Question}{\newpage \item \mbox{} \\}

% notation
\newcommand{\notation}[1]{
\textbf{Notation.} #1
}

% quantum gates
\newcommand{\qgate}[1]{\mathrm{#1}}
\newcommand{\SWAP}{\qgate{SWAP}}
\newcommand{\SSWAP}{\sqrt{\SWAP}}

% spaces
\newcommand{\lin}{\mathcal{L}}
\newcommand{\unitary}[1]{\mathcal{U}(#1)}
\newcommand{\density}{\mathcal{D}}
\newcommand{\stdbasis}{\mathcal{B}}

\author{Instructor: Mehmet Emre}
\date{CS 32 Fall '21}
\title{Homework 11: Inheritance and Derived Classes}
\hypersetup{
 pdfauthor={Instructor: Mehmet Emre},
 pdftitle={Homework 11: Inheritance and Derived Classes},
 pdfkeywords={},
 pdfsubject={},
 pdfcreator={Emacs 27.2 (Org mode 9.4.4)}, 
 pdflang={English}}
\begin{document}

\maketitle
\textbf{Due: 11/2 2pm} \\ 
\vspace{1em}
\textbf{Name \& Perm \#: Bharat Kathi (5938444)} \\ 
\textbf{Homework buddy (leave blank if you worked alone):}

\textbf{Reading:} PS 15.1, 15.2, DS 14.1

\section{}
\label{sec:orgf1389b4}

For each statement, indicate if it is True or False by circling T or F.
If you need to cross out an answer, be sure that your final answer is
clear and unambigous---otherwise it will receive no credit.

\begin{center}
\begin{tabular}{llll}
(3 pts) & An object of a derived class has access to the public methods of its base class & T & \\
(3 pts) & An object of a base class has access to the private helper methods of its derived class &  & F\\
(3 pts) & Destructors are not inherited by derived classes & T & \\
(3 pts) & An object of a derived class inherits the copy constructor of its base class &  & F\\
(3 pts) & Operators are passed down inheritance hierarchies & T & \\
(3 pts) & Destructors in derived classes are called after their base class calls its destructor &  & F\\
(3 pts) & Constructors of base classes are accessible by derived classes & T & \\
\end{tabular}
\end{center}

\section{}
\label{sec:org15ab71b}

Assume there is a class called \textbf{Student} that has private member variables
\texttt{string name, int perm}. Assume that Student has getters and setters for each of
these data members \texttt{getName}, \texttt{setName}, \texttt{getPerm} and \texttt{setPerm}, and a
constructor that takes \texttt{name} and \texttt{perm} as parameters.  Assume that the class
is declare in file \texttt{student.h} . In other words, you are given the following
class definition:

\begin{verbatim}
class Student {
  std::string name;
  int perm;
public:
  Student(std::string name, int perm) :
    name(name), perm(perm) {}

  const std::string & getName() const {
    return name;
  }

  void setName(std::string name) {
    this->name = name;
  }

  int getPerm() const {
    return perm;
  }

  void setPerm(int perm) {
    this->perm = perm;
  }
}
\end{verbatim}

\begin{enumerate}
\item (8 pts) Write the contents of a .h file for a derived a class called
\textbf{CmpscStudent} that inherits from \textbf{Student} and has additional data members
including \texttt{*string* ugradDegreeType, *bool* graduateStudent}.  Include
prototypes for a public constructor that initializes all of the data members
of \textbf{CmpscStudent} as well as getters (but not setters) for the additional
data members of CmpscStudent. Do not implement these constructors and methods
as inline. Just give the prototypes in the class definition.

\begin{description}
  \item[Answer:] .\\
  \begin{verbatim}
    class CmpscStudent: public Student {
      std::string ugradDegreeType;
      bool graduateStudent;
    public:
       CmpscStudent(
          string name,
          int perm,
          string ugradDegreeType,
          bool graduateStudent
       );
       string getUgradDegreeType();
       bool isGraduateStudent();
    };
    \end{verbatim}
\end{description}

\newpage

\item (8 pts) Write the function definition for the constructor
\textbf{CmpscStudent} that takes as its parameters all of the data members
of CmpscStudent, and fully initializes the object being constructed,
as it would appears in the .cpp file.

\begin{description}
  \item[Answer:] .\\
  \begin{verbatim}
       CmpscStudent::CmpscStudent(
          string name,
          int perm,
          string ugradDegreeType,
          bool graduateStudent
       ) : Student(name, perm) {
         this->ugradDegreeType = ugradDegreeType;
         this->graduateStudent = graduateStudent;
       }
    \end{verbatim}
\end{description}

\end{enumerate}
\end{document}