% Created 2022-04-04 Mon 15:02
% Intended LaTeX compiler: pdflatex
\documentclass[11pt]{article}
\usepackage[utf8]{inputenc}
\usepackage[T1]{fontenc}
\usepackage{graphicx}
\usepackage{grffile}
\usepackage{longtable}
\usepackage{wrapfig}
\usepackage{rotating}
\usepackage[normalem]{ulem}
\usepackage{amsmath}
\usepackage{textcomp}
\usepackage{amssymb}
\usepackage{capt-of}
\usepackage{hyperref}
% org has these:
% \usepackage[T1]{fontenc}
% \usepackage[utf8]{inputenc}

\usepackage[letterpaper,margin=2cm]{geometry}
%\usepackage{fullpage}
% a better font family
\usepackage{lmodern} % for sans serif, the fonts below overrride this
\usepackage{eulervm}
%\usepackage{newpxtext}
% \usepackage{newpxmath}
\usepackage{mathpazo}
%\usepackage{beton}
%\usepackage[default]{comicneue}

\usepackage{xcolor}

% \usepackage{amssymb}
\usepackage{amsthm}
% \usepackage{amsmath}
\usepackage{amstext}
% math tools for amsmath
\usepackage{mathtools}
% ceiling and floor symbols
\DeclarePairedDelimiter\ceil{\lceil}{\rceil}
\DeclarePairedDelimiter\floor{\lfloor}{\rfloor}

\usepackage{braket} % to enter bra-ket notation - easily
\usepackage{wasysym}

% to draw quantum circuits
\usepackage{qcircuit}

% for more enumeration style options
% \usepackage{enumerate}
\usepackage{enumitem}

% for references, org-mode already has these
% \usepackage{varioref}
% \usepackage[hidelinks]{hyperref}
\usepackage{cleveref} % for automatically entering reference types (theorem, figure, section etc.)

\usepackage{titling, titlesec}
\usepackage{slantsc} 

% bibliography style
\bibliographystyle{acm}

% theorem types
\newtheorem{thm}{Theorem}
\newtheorem{cor}[thm]{Corollary}
\newtheorem{lem}[thm]{Lemma}
% theorem types' names for cleveref
\crefname{thm}{theorem}{theorems}
\crefname{lem}{lemma}{lemmas}
\crefname{cor}{corollary}{corollaries}
\Crefname{thm}{Theorem}{Theorems}
\Crefname{lem}{Lemma}{Lemmas}
\Crefname{cor}{Corollary}{Corollaries}

\newcommand{\draftnote}[1]{\textcolor{red}{#1}}

\newcommand*{\QED}{\null\hfill\qedsymbol}

% I mistype texttt sometimes
\newcommand{\textt}{\texttt}
% ease to enter some functions in math mode
\newcommand{\fn}{\mathrm}
\newcommand{\bigO}{\fn{O}}
\newcommand{\argmax}{\fn{argmax}}
\newcommand{\ii}{\imath}
\newcommand{\qre}{q^\circ}
\newcommand{\qim}{q^\ast}
\newcommand{\transop}{\mathcal{E}}
\newcommand{\lcm}{\fn{lcm}}
\newcommand{\Str}{\mathbb{S}}
\newcommand{\Sign}{\Str}
\newcommand{\Concrete}{\mathcal{P}(\Sigma^*)}
\newcommand{\true}{\mathrm{true}}
\newcommand{\false}{\mathrm{false}}
\newcommand{\conj}{\overline}
\newcommand{\otherwise}{\textrm{ otherwise}}
\newcommand{\where}[1]{\textrm{ where {#1}}}

% norms and absolute values
\newcommand{\abs}[1]{\left\vert{}#1\right\vert}
\newcommand{\norm}[1]{\left\Vert{}#1\right\Vert}
\newcommand{\fnorm}[1]{\norm{#1}_F}
\newcommand{\pnorm}[2]{\norm{#1}_{#2}}
\newcommand{\supx}[1]{\sup_{#1 \neq 0}}
\newcommand{\supn}[2]{\sup_{\norm{#1}_{#2} = 1}}

% trace
\DeclareMathOperator{\tr}{Tr}

%\newcommand{\span}{\fn{span}}
\newcommand{\cols}{\fn{cols}}
\newcommand{\range}{\mathcal{R}}
\newcommand{\rank}{\fn{rank}}
\newcommand{\eye}{\fn{I}}

\newcommand{\diag}[1]{\fn{diag}\left\{#1\right\}}

% small matrices
\newenvironment{mat}[1]{\left(\begin{array}{#1}}{\end{array}\right)}
\newcommand{\matone}[1]{\begin{mat}{l}#1\end{mat}}
\newcommand{\mattwo}[1]{\begin{mat}{ll}#1\end{mat}}
\newcommand{\matthree}[1]{\begin{mat}{lll}#1\end{mat}}
\newcommand{\matfour}[1]{\begin{mat}{llll}#1\end{mat}}

\newcommand{\eyetwo}{\mattwo{1 & 0 \\ 0 & 1}}

% extra functions
\DeclareMathOperator{\fid}{F} % fidelity
\DeclareMathOperator{\vecOf}{vec} % vectorization

% special sets
\newcommand{\mset}{\mathbb}
\newcommand{\reals}{\mset{R}}
\newcommand{\realmat}[2]{\reals^{#1 \times #2}}
\newcommand{\complex}{\mset{C}}
\newcommand{\complexmat}[2]{\complex^{#1 \times #2}}
\newcommand{\zahlen}{\mset{Z}}
\newcommand{\ints}{\zahlen}
\newcommand{\nats}{\mset{N}}
% group, field etc.
\newcommand{\field}{\mset{F}}

% questions
\newenvironment{questions}
{\begin{enumerate}[label={Question \arabic*},wide,font=\bf]}
  {\end{enumerate}}

\newcommand{\question}{\item \mbox{} \\}
\newcommand{\Question}{\newpage \item \mbox{} \\}

% notation
\newcommand{\notation}[1]{
\textbf{Notation.} #1
}

% quantum gates
\newcommand{\qgate}[1]{\mathrm{#1}}
\newcommand{\SWAP}{\qgate{SWAP}}
\newcommand{\SSWAP}{\sqrt{\SWAP}}

% spaces
\newcommand{\lin}{\mathcal{L}}
\newcommand{\unitary}[1]{\mathcal{U}(#1)}
\newcommand{\density}{\mathcal{D}}
\newcommand{\stdbasis}{\mathcal{B}}

\author{Instructor: Mehmet Emre}
\date{CS 32 Spring '22}
\title{Homework 4: Abstract Data Types}
\hypersetup{
 pdfauthor={Instructor: Mehmet Emre},
 pdftitle={Homework 4: Abstract Data Types},
 pdfkeywords={},
 pdfsubject={},
 pdfcreator={Emacs 27.2 (Org mode 9.4.4)}, 
 pdflang={English}}
\begin{document}

\maketitle
\textbf{Due: 4/13 12:30pm} \\ 
\vspace{1em}
\textbf{Name \& Perm \#: Bharat Kathi (5938444)} \\ 
\textbf{Homework buddy (leave blank if you worked alone):}

\vspace{1em}
\textbf{Reading:} "Abstract Data Types", PS 10.3


\section{}
\label{sec:orgf0d6776}

Savitch makes the following observations in the introduction to
Section 10.3:


\begin{quote}
"Unless \ldots{} defined and used with care, programmer-defined types can be
used in unintuitive ways that make a program difficult to understand and
difficult to modify. The best way to avoid these problems is to make
sure all the data types you define are ADTs. The way that you do this in
C++ is to use classes, but not every class is an ADT. To make it an ADT
you must define the class in a certain way\ldots{}"
\end{quote}

\begin{enumerate}
\item (4 pts) What does ADT stand for?
\begin{description}
    \item[Answer:] .\\
    ADT stands for Abstract Data Types.
\end{description}
\item (4 pts) According to Savitch, a data type consists of a collection
of values, together with \ldots{} what?
\begin{description}
    \item[Answer:] .\\
    Data types consist of selections of values together with sets of basic operations for those values.
\end{description}
\item (4 pts) According to Savitch, a data type is called an ADT if
\ldots{} what?
\begin{description}
    \item[Answer:] .\\
    A data type is an ADT if the actual implementation of values and operations are abstracted to the programmer.
\end{description}
\end{enumerate}

\section{}
\label{sec:orge0ca3b4}

Savitch describes two ways of characterizing what makes a class an ADT:
first he describes two things that should be completely separated, and
then he describes three rules for achieving that separation.

What are the two things that should be separated?

\begin{enumerate}
\item (4 pts) First thing:
\begin{description}
    \item[Answer:] .\\
    The specification for how the type is used by the programmer.
\end{description}
\item (4 pts) Second thing:
\begin{description}
    \item[Answer:] .\\
     The details of how the type is implementated.
\end{description}
\end{enumerate}

\section{}
\label{sec:org98e459a}
What are the three rules for keeping those things separate?

\begin{enumerate}
\item (4 pts) Rule 1:
\begin{description}
    \item[Answer:] .\\
    All member variables should be private.
\end{description}
\item (4 pts) Rule 2:
\begin{description}
    \item[Answer:] .\\
    All basic operations should be public and should explain to the programmer its use cases.
\end{description}
\item (4 pts) Rule 3:
\begin{description}
    \item[Answer:] .\\
    All helper functions should be private.
\end{description}
\end{enumerate}

\section{(4 pts)}
\label{sec:orgec4222c}

When you define an ADT as a class, what items are considered part of
the interface for the ADT?
\begin{description}
    \item[Answer:] .\\
    The interface for the ADT consists of all the parts of the class that are required to use that data type (e.g. public member functions).
\end{description}

\section{(4 pts)}
\label{sec:orgb376198}

When you define an ADT as a class, what items are considered part of
the implementation for the ADT?
\begin{description}
    \item[Answer:] .\\
    The implementation of an ADT consists of all the parts of code that actually define the interface. This includes the definitions of public and private member functions, as well as private members of the class.
\end{description}
\end{document}