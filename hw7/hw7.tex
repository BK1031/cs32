% Created 2022-04-18 Mon 05:05
% Intended LaTeX compiler: pdflatex
\documentclass[11pt]{article}
\usepackage[utf8]{inputenc}
\usepackage[T1]{fontenc}
\usepackage{graphicx}
\usepackage{longtable}
\usepackage{wrapfig}
\usepackage{rotating}
\usepackage[normalem]{ulem}
\usepackage{amsmath}
\usepackage{amssymb}
\usepackage{capt-of}
\usepackage{hyperref}
% org has these:
% \usepackage[T1]{fontenc}
% \usepackage[utf8]{inputenc}

\usepackage[letterpaper,margin=2cm]{geometry}
%\usepackage{fullpage}
% a better font family
\usepackage{lmodern} % for sans serif, the fonts below overrride this
\usepackage{eulervm}
%\usepackage{newpxtext}
% \usepackage{newpxmath}
\usepackage{mathpazo}
%\usepackage{beton}
%\usepackage[default]{comicneue}

\usepackage{xcolor}

% \usepackage{amssymb}
\usepackage{amsthm}
% \usepackage{amsmath}
\usepackage{amstext}
% math tools for amsmath
\usepackage{mathtools}
% ceiling and floor symbols
\DeclarePairedDelimiter\ceil{\lceil}{\rceil}
\DeclarePairedDelimiter\floor{\lfloor}{\rfloor}

\usepackage{braket} % to enter bra-ket notation - easily
\usepackage{wasysym}

% to draw quantum circuits
\usepackage{qcircuit}

% for more enumeration style options
% \usepackage{enumerate}
\usepackage{enumitem}

% for references, org-mode already has these
% \usepackage{varioref}
% \usepackage[hidelinks]{hyperref}
\usepackage{cleveref} % for automatically entering reference types (theorem, figure, section etc.)

\usepackage{titling, titlesec}
\usepackage{slantsc} 

% bibliography style
\bibliographystyle{acm}

% theorem types
\newtheorem{thm}{Theorem}
\newtheorem{cor}[thm]{Corollary}
\newtheorem{lem}[thm]{Lemma}
% theorem types' names for cleveref
\crefname{thm}{theorem}{theorems}
\crefname{lem}{lemma}{lemmas}
\crefname{cor}{corollary}{corollaries}
\Crefname{thm}{Theorem}{Theorems}
\Crefname{lem}{Lemma}{Lemmas}
\Crefname{cor}{Corollary}{Corollaries}

\newcommand{\draftnote}[1]{\textcolor{red}{#1}}

\newcommand*{\QED}{\null\hfill\qedsymbol}

% I mistype texttt sometimes
\newcommand{\textt}{\texttt}
% ease to enter some functions in math mode
\newcommand{\fn}{\mathrm}
\newcommand{\bigO}{\fn{O}}
\newcommand{\argmax}{\fn{argmax}}
\newcommand{\ii}{\imath}
\newcommand{\qre}{q^\circ}
\newcommand{\qim}{q^\ast}
\newcommand{\transop}{\mathcal{E}}
\newcommand{\lcm}{\fn{lcm}}
\newcommand{\Str}{\mathbb{S}}
\newcommand{\Sign}{\Str}
\newcommand{\Concrete}{\mathcal{P}(\Sigma^*)}
\newcommand{\true}{\mathrm{true}}
\newcommand{\false}{\mathrm{false}}
\newcommand{\conj}{\overline}
\newcommand{\otherwise}{\textrm{ otherwise}}
\newcommand{\where}[1]{\textrm{ where {#1}}}

% norms and absolute values
\newcommand{\abs}[1]{\left\vert{}#1\right\vert}
\newcommand{\norm}[1]{\left\Vert{}#1\right\Vert}
\newcommand{\fnorm}[1]{\norm{#1}_F}
\newcommand{\pnorm}[2]{\norm{#1}_{#2}}
\newcommand{\supx}[1]{\sup_{#1 \neq 0}}
\newcommand{\supn}[2]{\sup_{\norm{#1}_{#2} = 1}}

% trace
\DeclareMathOperator{\tr}{Tr}

%\newcommand{\span}{\fn{span}}
\newcommand{\cols}{\fn{cols}}
\newcommand{\range}{\mathcal{R}}
\newcommand{\rank}{\fn{rank}}
\newcommand{\eye}{\fn{I}}

\newcommand{\diag}[1]{\fn{diag}\left\{#1\right\}}

% small matrices
\newenvironment{mat}[1]{\left(\begin{array}{#1}}{\end{array}\right)}
\newcommand{\matone}[1]{\begin{mat}{l}#1\end{mat}}
\newcommand{\mattwo}[1]{\begin{mat}{ll}#1\end{mat}}
\newcommand{\matthree}[1]{\begin{mat}{lll}#1\end{mat}}
\newcommand{\matfour}[1]{\begin{mat}{llll}#1\end{mat}}

\newcommand{\eyetwo}{\mattwo{1 & 0 \\ 0 & 1}}

% extra functions
\DeclareMathOperator{\fid}{F} % fidelity
\DeclareMathOperator{\vecOf}{vec} % vectorization

% special sets
\newcommand{\mset}{\mathbb}
\newcommand{\reals}{\mset{R}}
\newcommand{\realmat}[2]{\reals^{#1 \times #2}}
\newcommand{\complex}{\mset{C}}
\newcommand{\complexmat}[2]{\complex^{#1 \times #2}}
\newcommand{\zahlen}{\mset{Z}}
\newcommand{\ints}{\zahlen}
\newcommand{\nats}{\mset{N}}
% group, field etc.
\newcommand{\field}{\mset{F}}

% questions
\newenvironment{questions}
{\begin{enumerate}[label={Question \arabic*},wide,font=\bf]}
  {\end{enumerate}}

\newcommand{\question}{\item \mbox{} \\}
\newcommand{\Question}{\newpage \item \mbox{} \\}

% notation
\newcommand{\notation}[1]{
\textbf{Notation.} #1
}

% quantum gates
\newcommand{\qgate}[1]{\mathrm{#1}}
\newcommand{\SWAP}{\qgate{SWAP}}
\newcommand{\SSWAP}{\sqrt{\SWAP}}

% spaces
\newcommand{\lin}{\mathcal{L}}
\newcommand{\unitary}[1]{\mathcal{U}(#1)}
\newcommand{\density}{\mathcal{D}}
\newcommand{\stdbasis}{\mathcal{B}}

\author{Instructor: Mehmet Emre}
\date{CS 32 Spring '22}
\title{Homework 7: Sorting}
\hypersetup{
 pdfauthor={Instructor: Mehmet Emre},
 pdftitle={Homework 7: Sorting},
 pdfkeywords={},
 pdfsubject={},
 pdfcreator={Emacs 28.1 (Org mode 9.5.2)}, 
 pdflang={English}}
\begin{document}

\maketitle
\textbf{Due: 05/04 12:30pm} \\ 
\vspace{1em}
\textbf{Name \& Perm \# (no partners allowed): Bharat Kathi (5938444)} \\ 
\vspace{1em}
\textbf{Reading:} DS 13.1, also review DS 12.1, 2.6, 6.1


Please also read the handout at \url{http://cs.ucsb.edu/\~richert/cs32/misc/h07-handout.pdf}

\section{(10 pts)}
\label{sec:org60740aa}
\textbf{Briefly} explain: What does it mean to say that an algorithm has \emph{quadratic}
worst-case run time?

\begin{description}
    \item[Answer:] This means that the Big O notation for the worst-case runtime is $O(n^2)$. Specifically, it means that for the given $n$ inputs, the algorithm will perform some factorm of $n^2$ operations in its worst-case execution.
\end{description}

\section{}
\label{sec:org9fa5c4e}
Show the steps of bubble sort following the example of the solved problems on
the handout for the algorithm and the format of your answer. \textbf{Stop after the
first pass with no swaps.}

\subsection{5 pts}
\label{sec:orgac34f58}

\begin{center}
\begin{tabular}{lrrrrr}
initial values & 8 & 5 & 3 & 10 & 2\\
\hline
i = 4 & 5 & 3 & 8 & 2 & 10\\
\hline
i = 3 & 3 & 5 & 2 & 8 & 10\\
\hline
i = 2 & 3 & 2 & 5 & 8 & 10\\
\hline
i = 1 & 2 & 3 & 5 & 8 & 10\\
\end{tabular}
\end{center}

\subsection{5 pts}
\label{sec:orgeb32a4b}

\begin{center}
\begin{tabular}{lrrrrr}
initial values & 40 & 42 & -3 & 10 & 0\\
\hline
i = 4 & 40 & -3 & 10 & 0 & 42\\
\hline
i = 3 & -3 & 10 & 0 & 40 & 42\\
\hline
i = 2 & -3 & 0 & 10 & 40 & 42\\
\hline
i = 1 & -3 & 0 & 10 & 40 & 42\\
\end{tabular}
\end{center}


\section{}
\label{sec:org0da6c76}
Show the steps of \textbf{insertion sort} \emph{following the example of the solved
problems on the handout} for the algorithm and the format of your answer.


\subsection{5 pts}
\label{sec:org353c073}

\begin{center}
\begin{tabular}{lrrrrr}
initial values & 8 & 5 & 3 & 10 & 2\\
\hline
i = 0 & 10 & 8 & 5 & 3 & 2\\
\hline
i = 1 & 8 & 10 & 5 & 3 & 2\\
\hline
i = 2 & 5 & 8 & 10 & 3 & 2\\
\hline
i = 3 & 3 & 5 & 8 & 10 & 2\\
\hline
i = 4 & 2 & 3 & 5 & 8 & 10\\
\end{tabular}
\end{center}

\subsection{5 pts}
\label{sec:orga2d4b75}

\begin{center}
\begin{tabular}{lrrrrr}
initial values & 40 & 42 & -3 & 10 & 0\\
\hline
i = 0 & 40 & 42 & -3 & 10 & 0\\
\hline
i = 1 & 40 & 42 & -3 & 10 & 0\\
\hline
i = 2 & -3 & 40 & 42 & 10 & 0\\
\hline
i = 3 & -3 & 10 & 40 & 42 & 0\\
\hline
i = 4 & -3 & 0 & 10 & 40 & 42\\
\end{tabular}
\end{center}

\section{}
\label{sec:org9953340}
Show the steps of \textbf{selection sort} \emph{following the example of the solved
problems on the handout} for the algorithm and the format of your answer.
\textbf{Show all rows even for passes that no swaps occur.}

\subsection{5 pts}
\label{sec:org715eabd}

\begin{center}
\begin{tabular}{lrrrrr}
initial values & 8 & 5 & 3 & 10 & 2\\
\hline
i = 4 & 8 & 5 & 3 & 2 & 10\\
\hline
i = 3 & 2 & 5 & 3 & 8 & 10\\
\hline
i = 2 & 2 & 3 & 5 & 8 & 10\\
\hline
i = 1 & 2 & 3 & 5 & 8 & 10\\
\end{tabular}
\end{center}

\subsection{5 pts}
\label{sec:org64dd6f6}

\begin{center}
\begin{tabular}{lrrrrr}
initial values & 40 & 42 & -3 & 10 & 0\\
\hline
i = 4 & 40 & 0 & -3 & 10 & 42\\
\hline
i = 3 & 10 & 0 & -3 & 40 & 42\\
\hline
i = 2 & -3 & 0 & 10 & 40 & 42\\
\hline
i = 1 & -3 & 0 & 10 & 40 & 42\\
\end{tabular}
\end{center}
\end{document}