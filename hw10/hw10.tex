% Created 2022-04-30 Sat 17:49
% Intended LaTeX compiler: pdflatex
\documentclass[11pt]{article}
\usepackage[utf8]{inputenc}
\usepackage[T1]{fontenc}
\usepackage{graphicx}
\usepackage{longtable}
\usepackage{wrapfig}
\usepackage{rotating}
\usepackage[normalem]{ulem}
\usepackage{amsmath}
\usepackage{amssymb}
\usepackage{capt-of}
\usepackage{hyperref}
% org has these:
% \usepackage[T1]{fontenc}
% \usepackage[utf8]{inputenc}

\usepackage[letterpaper,margin=2cm]{geometry}
%\usepackage{fullpage}
% a better font family
\usepackage{lmodern} % for sans serif, the fonts below overrride this
\usepackage{eulervm}
%\usepackage{newpxtext}
% \usepackage{newpxmath}
\usepackage{mathpazo}
%\usepackage{beton}
%\usepackage[default]{comicneue}

\usepackage{xcolor}

% \usepackage{amssymb}
\usepackage{amsthm}
% \usepackage{amsmath}
\usepackage{amstext}
% math tools for amsmath
\usepackage{mathtools}
% ceiling and floor symbols
\DeclarePairedDelimiter\ceil{\lceil}{\rceil}
\DeclarePairedDelimiter\floor{\lfloor}{\rfloor}

\usepackage{braket} % to enter bra-ket notation - easily
\usepackage{wasysym}

% to draw quantum circuits
\usepackage{qcircuit}

% for more enumeration style options
% \usepackage{enumerate}
\usepackage{enumitem}

% for references, org-mode already has these
% \usepackage{varioref}
% \usepackage[hidelinks]{hyperref}
\usepackage{cleveref} % for automatically entering reference types (theorem, figure, section etc.)

\usepackage{titling, titlesec}
\usepackage{slantsc} 

% bibliography style
\bibliographystyle{acm}

% theorem types
\newtheorem{thm}{Theorem}
\newtheorem{cor}[thm]{Corollary}
\newtheorem{lem}[thm]{Lemma}
% theorem types' names for cleveref
\crefname{thm}{theorem}{theorems}
\crefname{lem}{lemma}{lemmas}
\crefname{cor}{corollary}{corollaries}
\Crefname{thm}{Theorem}{Theorems}
\Crefname{lem}{Lemma}{Lemmas}
\Crefname{cor}{Corollary}{Corollaries}

\newcommand{\draftnote}[1]{\textcolor{red}{#1}}

\newcommand*{\QED}{\null\hfill\qedsymbol}

% I mistype texttt sometimes
\newcommand{\textt}{\texttt}
% ease to enter some functions in math mode
\newcommand{\fn}{\mathrm}
\newcommand{\bigO}{\fn{O}}
\newcommand{\argmax}{\fn{argmax}}
\newcommand{\ii}{\imath}
\newcommand{\qre}{q^\circ}
\newcommand{\qim}{q^\ast}
\newcommand{\transop}{\mathcal{E}}
\newcommand{\lcm}{\fn{lcm}}
\newcommand{\Str}{\mathbb{S}}
\newcommand{\Sign}{\Str}
\newcommand{\Concrete}{\mathcal{P}(\Sigma^*)}
\newcommand{\true}{\mathrm{true}}
\newcommand{\false}{\mathrm{false}}
\newcommand{\conj}{\overline}
\newcommand{\otherwise}{\textrm{ otherwise}}
\newcommand{\where}[1]{\textrm{ where {#1}}}

% norms and absolute values
\newcommand{\abs}[1]{\left\vert{}#1\right\vert}
\newcommand{\norm}[1]{\left\Vert{}#1\right\Vert}
\newcommand{\fnorm}[1]{\norm{#1}_F}
\newcommand{\pnorm}[2]{\norm{#1}_{#2}}
\newcommand{\supx}[1]{\sup_{#1 \neq 0}}
\newcommand{\supn}[2]{\sup_{\norm{#1}_{#2} = 1}}

% trace
\DeclareMathOperator{\tr}{Tr}

%\newcommand{\span}{\fn{span}}
\newcommand{\cols}{\fn{cols}}
\newcommand{\range}{\mathcal{R}}
\newcommand{\rank}{\fn{rank}}
\newcommand{\eye}{\fn{I}}

\newcommand{\diag}[1]{\fn{diag}\left\{#1\right\}}

% small matrices
\newenvironment{mat}[1]{\left(\begin{array}{#1}}{\end{array}\right)}
\newcommand{\matone}[1]{\begin{mat}{l}#1\end{mat}}
\newcommand{\mattwo}[1]{\begin{mat}{ll}#1\end{mat}}
\newcommand{\matthree}[1]{\begin{mat}{lll}#1\end{mat}}
\newcommand{\matfour}[1]{\begin{mat}{llll}#1\end{mat}}

\newcommand{\eyetwo}{\mattwo{1 & 0 \\ 0 & 1}}

% extra functions
\DeclareMathOperator{\fid}{F} % fidelity
\DeclareMathOperator{\vecOf}{vec} % vectorization

% special sets
\newcommand{\mset}{\mathbb}
\newcommand{\reals}{\mset{R}}
\newcommand{\realmat}[2]{\reals^{#1 \times #2}}
\newcommand{\complex}{\mset{C}}
\newcommand{\complexmat}[2]{\complex^{#1 \times #2}}
\newcommand{\zahlen}{\mset{Z}}
\newcommand{\ints}{\zahlen}
\newcommand{\nats}{\mset{N}}
% group, field etc.
\newcommand{\field}{\mset{F}}

% questions
\newenvironment{questions}
{\begin{enumerate}[label={Question \arabic*},wide,font=\bf]}
  {\end{enumerate}}

\newcommand{\question}{\item \mbox{} \\}
\newcommand{\Question}{\newpage \item \mbox{} \\}

% notation
\newcommand{\notation}[1]{
\textbf{Notation.} #1
}

% quantum gates
\newcommand{\qgate}[1]{\mathrm{#1}}
\newcommand{\SWAP}{\qgate{SWAP}}
\newcommand{\SSWAP}{\sqrt{\SWAP}}

% spaces
\newcommand{\lin}{\mathcal{L}}
\newcommand{\unitary}[1]{\mathcal{U}(#1)}
\newcommand{\density}{\mathcal{D}}
\newcommand{\stdbasis}{\mathcal{B}}

\author{Instructor: Mehmet Emre}
\date{CS 32 Spring '22}
\title{Homework 10: Chained Hashing}
\hypersetup{
 pdfauthor={Instructor: Mehmet Emre},
 pdftitle={Homework 10: Chained Hashing},
 pdfkeywords={},
 pdfsubject={},
 pdfcreator={Emacs 28.1 (Org mode 9.5.2)}, 
 pdflang={English}}
\begin{document}

\maketitle
\textbf{Due: 5/11 12:30pm} \\ 
\vspace{1em}
\textbf{Name \& Perm \#: Bharat Kathi (5938444)} \\ 
\textbf{Homework buddy (leave blank if you worked alone):}

\textbf{Reading:} DS 12.3

Please also read the handout at
\url{https://cs.ucsb.edu/\~richert/cs32/misc/h09-handout.pdf} . 

\section{}
\label{sec:orgfd34428}
Hash functions are used when the main objective is to make searches as fast as
possible, i.e. an expected time O(1) search operation. Consider the typical
process of finding an element: an O(1) hash function H(x) computed on an item
x, giving some numerical location for a program to search for a value. A hash
collision in a chained hash table creates a linked list of all elements that
collide at any one index i.
\begin{enumerate}
\item (4 pts) The phrase \textbf{load factor} is a technical term that arises in
this context. Briefly: what does this term refer to?
\vspace{12em}
\item (4 pts) If something goes wrong, a chained hash strategy easily lead from an
average case O(1) search operation to average case O(n) search
operation. What is the circumstance in which things could go awry in this
way? In other words, what is the worst-case scenario for chained hashing?
\emph{Briefly} describe.
\vspace{12em}
\end{enumerate}

\newpage

\section{}
\label{sec:orgb45c31c}
Suppose that you are going to implement (without use of the STL) a chained
hashing scheme for objects of class Student, using a hash over the perm
number, using chained hashing. Suppose that a constant \texttt{ARRAY\_SIZE}
has been defined as the size of an array, and each element in that array
should be the head of a linked list of Students that hashes to the index of
that element.
\begin{enumerate}
\item (4 pts) Write a C++ definition for a struct that can
represent a node in each of these linked lists.
\vspace{8em}
\item (4 pts) Write a line of C++ that, if it appeared in the \texttt{private:} section
of the class definition for \texttt{class StudentHashTable}, would represent a
suitable array for the values, assuming that array was allocated on the heap
in the constructor of \texttt{class StudentHashtable}
\vspace{8em}
\item (8 pts) Write the definition of the constructor for
\texttt{StudentHashTable} as it would appear in the .cpp file. Be sure
to both allocate the space for the array on the heap, and set
all of the pointers in that array to null.
\vspace{10em}
\end{enumerate}

\section{(10 pts)}
\label{sec:org72c5b2a}
  Given the series of inputs [45, 7, 46, 90, 34, 6, 7, 71, 23, 24], insert each
of these integers into the chained hash table below, with a hash function
defined by \(H(x)= x \mod \mathit{tableSize}\). Draw in the linked lists to the
right of each array index, showing a pointer to null as the end of each
list. The Handout for this assignment has examples of the format we want for
your answer.

\begin{center}
\begin{tabular}{rl}
index & list (you may need to edit this to make it longer)\\
\hline
0 & \\
\hline
1 & \\
\hline
2 & \\
\hline
3 & \\
\hline
4 & \\
\end{tabular}
\end{center}
\end{document}